\documentclass[11pt]{article}

\usepackage{comment} % enables the use of multi-line comments (\ifx \fi) 
\usepackage[a4paper,margin=1cm]{geometry}
\usepackage[utf8]{inputenc}
\usepackage[ngerman]{isodate}
\usepackage{gensymb}
\usepackage{graphicx}
\usepackage{booktabs}% http://ctan.org/pkg/booktabs
\usepackage{tabularx}
\usepackage{ltablex} % Longtables with tabularx
\usepackage[x11names]{xcolor}
\usepackage{amsmath}
\usepackage{amssymb}
\usepackage{amsthm}
\usepackage{array}
\usepackage{wrapfig}
\usepackage{subcaption}
\usepackage{csquotes}
\usepackage{lscape}
\usepackage{geometry}
\usepackage{multicol}
\usepackage{bm}
\usepackage{enumitem}
\usepackage{hyperref}
\usepackage{mdframed}
\usepackage{scalerel}
\usepackage{stackengine}
\usepackage{mathtools}
\usepackage{pdfpages}

% Code highlighting
\usepackage{minted}
\surroundwithmdframed{minted}

% Be able to caption equations and float them in place
\usepackage{float}

\newmdtheoremenv{theorem}{Theorem}

\theoremstyle{definition}
\newmdtheoremenv{definition}{Definition}[section]

\geometry{a4paper, margin=2.4cm}

\newcommand\equalhat{\mathrel{\stackon[1.5pt]{=}{\stretchto{\scalerel*[\widthof{=}]{\wedge}{\rule{1ex}{3ex}}}{0.5ex}}}}
\newcommand\defeq{\mathrel{\overset{\makebox[0pt]{\mbox{\normalfont\tiny def}}}{=}}}
\newcolumntype{C}{>{\centering\arraybackslash}X}

\DeclarePairedDelimiter\abs{\lvert}{\rvert}
\DeclarePairedDelimiter\norm{\lVert}{\rVert}

\setcounter{tocdepth}{3}
\setcounter{secnumdepth}{3}

\graphicspath{{./img/}}

\begin{document}
	
\title{Private Law FS20}
\author{Pascal Baumann\\pascal.baumann@stud.hslu.ch}
\maketitle



For errors or improvement raise an issue or make a pull request on the \href{https://github.com/KilnOfTheSecondFlame/mse_summaries}{github repository}.

\tableofcontents
\newpage



\section{Introduction}
The goal of the module is to foster an understanding on the different dimensions of privacy and a thinking in the corresponding contexts. Transferring privacy aspects to
and within the private and work environment and reflecting upon these, establishing links with learning content in the MRU and the technical modules. Acquiring an appreciation of the legal aspects confronting an engineer in demanding professional situations. Gaining awareness in order to avoid damages due to infringements of rights or legal uncertainty. Acquisition of speaking and listening
skills in order to conduct the corresponding specialist discussions with experts.

Law is a social framework, that saves energy by providing an orientation how to manage conflicts and preserve the values of the culture the law was created in. The law legitimises public authorities and courts, and aims to control the power inherent in any system so that no imbalance is present. 

\subsection{Importance of Law in a Technical or Information Technology World}

The law provides a guideline of the allowed/maximum or obligatory requirement/minimum a system has to satisfy. It clarifies obligations, responsibilities and liability. Industry standards apply the law to a specific field, although they do not clear all legal questions most of the time.

In risk management not only technical or organisational risks have to be considered, but also legal measures to mitigate risks. It is prudent to consult legal as soon as possible in a project, otherwise it may be stopped at the last minute and be very costly. By law, management is responsible that an organisation complies to the legal framework.

\subsection{Importance of Privacy and Personal Data Protection}
Personal data in the wrong hands may harm the person in question in various forms. Privacy and the \emph{right to forget} is essential in personal development. An owner with a big collection of such personal data gets a lot of power with little oversight or control. One of the main tasks of a state is to protect its citizen, this includes protection from attacks on personal data.

\paragraph{One of the Most Vital Legal Questions}
\begin{definition}
	{\scriptsize (1)} \textbf{Who} wants from {\scriptsize (2)} \textbf{whom} {\scriptsize (3)}\textbf{what} based on what {\scriptsize (4)}\textbf{title} or right?
\end{definition}

\begin{figure}[H]
	\centering
	\includegraphics[width=0.8\linewidth]{img/conventions_moral_law}
	\caption{The relationship between conventions, morals and law}
	\label{fig:conventionsmorallaw}
\end{figure}

\subsection{Correct Legal Argumentation}
A statement or claim has to be justified by legal articles or arguments and the necessary evidence. Or is based on legal articles or arguments and with the necessary evidence arrives at a conclusion.

The law get classified by
\begin{itemize}
	\item Status: constitution, act, regulations or by-law
	\item Issuer: Federal, Cantonal and Communal Law
	\item Source of the Law: written law, common law, judicial tradition
	\item Involved Person: Civil Law or Public Law
\end{itemize}

At each of the federal, cantonal and communal level there is a separation of power into the \textbf{legislative}, the \textbf{executive} and the \textbf{Judiciary}, which provides a check and balance on the power of each authority.

\subsection{State, Cantons, Communes}

"Das Schweizervolk und die Kantone [...] bilden die Schweizerische Eidgenossenschaft", (1. Art BV).

"Der Kanton arbeitet mit den Gemeinden, den anderen Kantonen, dem Bund und, in seinem Zuständigkeitsbereich, mit dem Ausland zusammen." Therefore the State is only entitled to legislate and act in a territory or legal field where it has a constitutional legitimacy. The cantons are superior to the state in their power of legislation.

\begin{center}
	\includegraphics[width=0.4\linewidth,keepaspectratio]{law_hierarchy}
\end{center}

\begin{tabularx}{\linewidth}{X|X}
	\textbf{civil law} & \textbf{public law}\\
	\hline
	OR/ZGB & StGB\\
	GeBüV & FMG\\
	DSG & BÜPF/VÜPF\\
	URG, UWG & EIDI-V\\
	\vdots & \vdots
\end{tabularx}

Civil law is mastered by the \textbf{principle of freedom} of coalition and freedom of contract. Public law is mastered by the \textbf{principle of legality}, the control of power. This results in completely different jurisdictions with different procedures and rights for each.

\subsection{By-Law (Verordnung) $\neq$ Order (Verfügung)}

A By-Law is a \textbf{general, abstract regulation} as part of a law, while an order is an \textbf{individual, concrete application} of a law to a person.

\subsection{Legal Terms}
\begin{itemize}
	\item mandatory rules, stipulated rules and dispositive rules
	\item bona fide or good faith (ZGB 2)
	\item acting in good faith or a fair manner
	\item judicial discretion (ZGB 4)
	\item burden of proof (ZGB 8)
\end{itemize}

\subsection{International Law Framework}
Switzerland is integrated into the European legal framework by long-time traditions, unilateral and bilateral conventions. The IPRG ("Gesetz über das internationale Privatrecht") specifies the bridge between Swiss and foreign law. It rules which law is applicable and which court is competent. Parties can decide in most cases, under which jurisdiction they want to handle their disputes and which court will be competent.

There are generally three different instances in these jurisdictions:
\begin{itemize}
	\item Bezirksgericht
	\item Kantonsgericht
	\item Bundesgericht
\end{itemize}

\section{Copyright}
Through applying creative intellect a work is created. This work enjoys copyright, which means the creator may decide how this work is distributed. Copyright law guarantees them financial remuneration for the utilisation of their works. Moreover, the copyright law protects artists’ intellectual property in that they are able to defend themselves against
misappropriation of their work. Creative endeavour is a public good. Intellectual creations are the driving force of our economy.

\subsection{Prerequisites for Legal Protection of a Work}
A work must be an \textbf{intellectual creation} and have \textbf{Statistical Uniqueness}:
\begin{itemize}
	\item Individual character, characteristic features
	\item Sufficiently creative step, work is special and unique
\end{itemize}

There must be a perceptibility or expression of an idea. And ultimately a work must be created by humans.

With the new law all photographs and reproductions produced by similar methods of three-dimensional objects (film stills) will be subject to copyright protection.

\paragraph{Art 3 Swiss Copyright Act - Derivative Works}
\begin{enumerate}[label=\arabic* ]
	\item Derivative works are intellectual creations with an individual character that are \textbf{based upon pre-existing works, whereby the individual character of the latter remains identifiable.}
	\item Such works include, in particular, \textbf{translations} as well as audio-visual and other \textbf{adaptations}.
	\item Derivative works are protected as \textbf{works in their own right}.
	\item The protection of the works used in the derivative work remains reserved.
\end{enumerate}

To create a derivative work you require the consent of the copyright owner of the original work.

\subsection{Non-derivative Works}
\begin{enumerate}[label=\arabic* ]
	\item \textbf{rework}
	\begin{itemize}
		\item use of pre-existing works
		\item mere rework of an original work, no individual character
	\end{itemize}
	\item \textbf{new creation}
	\begin{itemize}
		\item original work is merely an inspiration and cannot be recognised or identified in the new creation
		\item the newly created work is protected individually in its own right
	\end{itemize}
\end{enumerate}

\subsection{Meaning of Copyright}
The \textbf{author of a copyrighted work} is always the natural person who created the work, which is known as \textbf{creator principle}.	Copyright law is an \textbf{absolute right} and thus excludes every other person. Copyrights can be transferred from the \textbf{original author} to another person or legal entity who then becomes the \textbf{right holder}.

\paragraph{Art 9 Swiss Copyright Act}
\begin{enumerate}[label=\arabic*]
	\item The author has the exclusive right to his own work and the right to recognition of his authorship.
	\item The author has the exclusive right to decide whether, when, how and under what author's designation his own work is published for the first time.
	\item A work is considered to be published when it has been made available for the first time by the author, or with his consent, to a large number of persons not constituting a private circle.
\end{enumerate}

\paragraph{Art 10 Swiss Copyright Act}
\begin{enumerate}[label=\arabic*]
	\item \textbf{The author has the exclusive right to decide whether, when and how his work is used.}
	\item The author has the right, in particular:
	\begin{enumerate}[label=\alph*.]
		\item to produce \textbf{copies} of the work, such as printed matter, phonograms, audio-visual fixations or data carriers;
		\item to offer, transfer or otherwise \textbf{distribute} copies of the work;
		\item to recite, \textbf{perform} or present a work, or make it \textbf{perceptible} somewhere else or \textbf{make it available} directly or through any kind of medium in such a way that persons may access it from a place and at a time individually chosen by them;
		\item to \textbf{broadcast} the work by radio, television or similar means, including by wire;
		\item to \textbf{retransmit} works by means of technical equipment, the provider of which is not the original broadcasting organisation, in particular including by wire;
		\item to make works made available, broadcast and retransmitted \textbf{perceptible}.
	\end{enumerate}
	\item The author of a computer program also has the exclusive rental right.
\end{enumerate}

\paragraph{Art 11 Swiss Copyright Act}
\begin{enumerate}[label=\arabic*]
	\item The author has the exclusive right to decide:
	\begin{enumerate}[label=\alph*.]
		\item whether, when and how the work may be \textbf{altered};
		\item whether, when and how the work may be used to create a \textbf{derivative work} or may be \textbf{included in a collected work}.
	\end{enumerate}
	\item Even where a third party is authorised by contract or law to alter the work or to use it to create a derivative work, the author may \textbf{oppose any distortion} of the work that is a violation of his personal rights.
	\item It is permissible to use existing works for the creation of parodies or other comparable variations on the work.
\end{enumerate}

\paragraph{Art 15 Swiss Copyright Act}
\begin{enumerate}[label=\arabic*]
	\item Where the owner of an original work of which no further copies exist has reason to assume that the author of the work has a legitimate interest in its preservation, he may not destroy the work without first offering to return it to the author. The owner may not request more than the material value of the work.
\end{enumerate}

\subsection{Transfer of Copyright}
Moral rights vest in the author and cannot be transferred or licensed, but they may be waived or transferred to heirs upon death.

\textbf{Economic rights may be transferred to third parties and transferred to heirs upon death, use rights may be licensed.}

The transfer of a work or copy of the work does not transfer the rights in the work, even if the original work is transferred.

\subsection{Joint Authorship}
This happen when multiple people \textbf{work together} pursuant to a \textbf{common concept} to create a work together. They can decide jointly on what happens with the work and must all agree to that.

\subsection{Duration of Copyright}
In Switzerland there is an \textbf{automatic protection with no formalities required}. A work is protected under copyright as soon as it is \textbf{created}.

Copyright protection starts as soon as a work is created and the prerequisites
for protection are met. In Switzerland copyright protection expires \textbf{70 years after the death of the author}. The protection for \textbf{computer programs} ends \textbf{50 years after the death of the author}. The protection of \textbf{pictures which do not possess an individual character} ends \textbf{50 years after their creation}.

\subsection{Limitations of Copyright}
\begin{center}
	\includegraphics[width=0.8\linewidth]{img/limitations_of_copyright}
\end{center}
Private use includes only the \textbf{use in a private setting for private purposes},
sharing with family and close friends (private circle). \textbf{Private use is not subject to a license fee}. The download of copyrighted works is considered private use. 

\subsection{Related Rights}
The Copyright Act also regulates related rights also referred to as neighbouring rights.
\begin{itemize}[label=-]
	\item rights of \textbf{performing artists} to their performances
	\item rights of \textbf{producers of phonograms and videograms} to their products
	\item rights of \textbf{broadcasters} to their radio and television broadcasts
\end{itemize}

Related rights protection expires \textbf{50 years after the performance}, the publication of the phonogram or videogram or the production of it, in the case that it is not published or the emission of the broadcast.

\subsection{Civil Proceedings}
A copyright claim between private persons or private legal entities are discussed in civil proceedings in a civil court. The \textbf{right holder can request from the civil court}:
\begin{itemize}
	\item a declaratory judgement, that is have a right established
	\item have an infringement banned or remedied
	\item have property confiscated
	\item have an order published
	\item claim damages
\end{itemize}

\subsection{Criminal Proceedings}

Criminal procedures may be initiated by the author / right holder within \textbf{three months} after becoming aware of an infringement or by the authorities, and may result in fines or imprisonment of up to one year (five years if done on a commercial basis).

\begin{itemize}
	\item use of a work within correctly indicating the author
	\item publishing a work
	\item change a work
	\item use for derivative work
	\item copies of works
	\item distributing or making works available
	\item broadcasting
	\item renting out of computer programs
\end{itemize}

\subsection{Collecting Societies}
The Federal Copyright Act is based on the view that the rights accruing to authors are essentially the \textbf{responsibility of the right holders to assert} for themselves.

The Federal Copyright Act only envisages collective management by collective rights management organisations in circumstances where \textbf{mass utilisation makes direct management virtually impossible}. To do so, the collecting societies establish tariffs which are negotiated with the primary user associations.

A comprehensive network of \textbf{reciprocity contracts with affiliated foreign societies} results in an \textbf{exchange} of rights and remunerations. The Swiss societies license the entire world repertoire.

\subsection{Copyright and the Internet}
Downloading, copying, scanning are types of use of a copyrighted work. This permission must be given \textbf{expressly} or \textbf{implied}. Links facilitates access to the page and \emph{implies} to the internet user that the linked pages are \textbf{associated}. Therefore, rights may be \textbf{infringed simply through linking}.

\section{Swiss Personal Data Protection}

\subsection{Reason for Data Protection}
\begin{minipage}{0.6\linewidth}
	Data Protection does not mean protecting data, but protecting people. Humans have a social (public) but also an individual nature (privacy). Human learning imply making faults, and an individual should not be reminded for his faults all his life, there needs to be a right to be forgotten.\\
	
	\vspace{1em}
	\noindent
	\textbf{Protection of Integrity / Personality - Art 28 CC (ZGB)} \\
	\noindent
	Any person whose personality rights are unlawfully infringed may petition the court for protecting against all those causing the infringement.
\end{minipage}
\begin{minipage}{0.4\linewidth}
	\begin{center}
		\includegraphics[width=0.8\linewidth]{img/spheres.png}
	\end{center}
\end{minipage}

This means that any editing of personal data is \textbf{nota bene illegal}. Unless it is justified by the consent of the person whose rights are infringed or by law or by an overriding private or public interest.

\subsubsection{Legal Bases}
\begin{itemize}[label=-]
	\item Swiss Constituion Art 13
	\item Swiss Federal Act on Data Protection
	\item Ordinance to FADP (Verordnung zum Bundesgesetz über den Datenschutz, VDSG)
	\item EU-GDPR (General Data Protection Regulation - DSGVO)
\end{itemize}

\subsection{Changes in the New Data Protection Law (eDSG)}

\begin{itemize}[label=-]
	\item Only for natural entities (Art 2 eDSG)
	\item Extension of the territorial validity, abroad also (Art 2a eDSG)
	\item Naming a representative in Switzerland (Art 12 eDSG)
	\item Extention of Profiling (Art 4 lit f eDSG)
	\item Data Protection Consultant (Art 9 eDSG)
	\item List of processing activities (Art 11 eDSG)
	\item Data protection impact assessment (Art 20 eDSG)
	\item Reporting data security breaches (Art 22 eDSG)
	\item Legal claim for deletion of personal data (Art 28 Abs 2 lit c eDSG)
	\item Fines of up to 250'000 CHF are imposed on private persons who deliberately… (Art 55 eDSG)
\end{itemize}

\subsection{Area of Application Art 2 DSG}
\begin{enumerate}[label=\arabic*]
	\item This Act applies to the processing of data pertaining to natural persons and legal persons by:
	\begin{enumerate}[label=\alph*.]
		\item \textbf{private persons};
		\item federal bodies.
	\end{enumerate}
	\item \textbf{It does not apply to}:
	\begin{enumerate}[label=\alph*.]
		\item personal data that is \textbf{processed by a natural person exclusively for personal use and which is not disclosed to outsiders};
		\item deliberations of the Federal Assembly and in parliamentary committees;
		\item pending civil proceedings, criminal proceedings, international mutual assistance proceedings and proceedings under constitutional or under administrative law, with the exception of administrative proceedings of first instance;
		\item public registers based on private law;
		\item personal data processed by the International Committee of the Red Cross.
	\end{enumerate}
\end{enumerate}

\subsection{Terms and Definitions Art 3 DSG}
\begin{itemize}[label=]
	\item The following definitions apply:
	\begin{enumerate}[label=\alph*.]
		\item \textbf{personal data (data)}: all information relating to an identified or identifiable person;
		\item \textbf{data subjects}: natural or legal persons whose data is processed;
		\item \textbf{sensitive personal data}: data on:
		\begin{enumerate}[label=\arabic*.]
			\item religious, ideological, political or trade union-related views or activities,
			\item health, the intimate sphere or the racial origin,
			\item social security measures,
			\item administrative or criminal proceedings and sanctions;
		\end{enumerate}
		\item \textbf{personality profile}: a collection of data that permits an assessment of essential characteristics of the personality of a natural person;
		\item \textbf{processing}: any operation with personal data, irrespective of the means applied and the procedure, and in particular the collection, storage, use, revision, disclosure, archiving or destruction of data;
		\item \textbf{disclosure}: making personal data accessible, for example by permitting access, transmission or publication;
		\item \textbf{data file}: any set of personal data that is structured in such a way that the data is accessible by data subject;
		\item \textbf{federal bodies}: federal authorities and services as well as persons who are entrusted with federal public tasks;
		\item \textbf{controller of the data file}: private persons or federal bodies that decide on the purpose and content of a data file;
	\end{enumerate}
\end{itemize}

\subsection{Principles Art 4 DSG}
\begin{enumerate}[label=\arabic*]
	\item Personal data may only be processed \underline{lawfully}.
	\item Its processing must be carried out \underline{in good faith} and \underline{must be proportionate}.
	\item Personal data may only be processed \underline{for the purpose indicated at the time of collection}, that is evident from the circumstances, or that is provided for by law.
	\item The collection of personal data and in particular the purpose of its processing \underline{must be evident} to the data subject.
	\item If the consent of the data subject is required for the processing of personal data, such consent \underline{is valid only if given voluntarily} on the provision of adequate information. Additionally, consent must be given \underline{expressly} in the case of processing of sensitive personal data or personality profiles.
\end{enumerate}

\subsection{Accuracy of Personal Data Art 5 DSG}
\begin{enumerate}[label=\arabic*]
	\item Anyone who processes personal data \textbf{must make certain that it is correct}. He must take all reasonable measures to ensure that data that is incorrect or incomplete in view of the purpose of its collection is either \textbf{corrected or destroyed}. 
	\item \textbf{Any data subject may request that incorrect data be corrected}.
\end{enumerate}

\subsection{Cross-Border Disclosure Art 6 DSG}
\begin{enumerate}[label=\arabic*]
	\item Personal data \textbf{may not be disclosed abroad} if the privacy of the data subjects \textbf{would be seriously endangered thereby}, in particular due to the absence of legislation that guarantees adequate protection.
	\item In the absence of legislation that guarantees adequate protection, personal data may be disclosed abroad only if: ...
\end{enumerate}

\subsection{Accuracy of Personal Data Art 7 DSG}
\begin{enumerate}[label=\arabic*]
	\item Personal data must be protected against \textbf{unauthorised processing} through \textbf{adequate technical and organisational measures}.
	\item The Federal Council issues detailed provisions on the minimum standards for data security.
\end{enumerate}

\subsection{Right to Information Art 8 DSG}
\begin{enumerate}[label=\arabic*]
	\item Any person may request information from the controller of a data file as to whether data concerning them is being processed.
	\item The controller of a data file \textbf{must notify the data subject}:
	\begin{enumerate}[label=\alph*.]
		\item \textbf{of all available data} concerning the subject in the data file, including the available information on the \textbf{source of the data};
		\item \textbf{the purpose of} and if applicable \textbf{the legal basis} for the processing as well as the \textbf{categories of the personal data} processed, the other parties involved with the file and \textbf{the data recipient}.
	\end{enumerate}
	\item The controller of a data file may arrange for data on the health of the data subject to be communicated by a doctor designated by the subject.
	\item If the controller of a data file has personal data processed by a third party, \textbf{the controller remains under an obligation to provide information}. The third party is under an obligation to provide information if he does not disclose the identity of the controller or if the controller is not domiciled in Switzerland.
	\item The information must normally be provided in \textbf{writing, in the form of a printout or a photocopy, and is free of charge}. The Federal Council regulates exceptions.
	\item \textbf{No one may waive the right to information in advance}.
\end{enumerate}

\subsection{Right to be Informed}
\begin{itemize}
	\item \textbf{All judicious persons} are authorised
	\item An \textbf{owner of data collection} is legally obligated
	\item There is \textbf{no} form of request
	\item The range is \textbf{all} of the personal information
	\item Deadline to inform the requester \textbf{in written form or upon consent per insight within 30 days generally}
	\item \textbf{Generally free of charge}
\end{itemize}

\subsection{Duty To Provide Information On The Collection Of Sensitive Personal Data And Personality Profiles Art 14 DSG}
\begin{enumerate}[label=\arabic*]
	\item The controller of the data file \textbf{is obliged to inform the data subject} of the collection \textbf{of sensitive personal data or personality profiles}; this duty to provide information also applies where the data is collected from third parties.
	\item The data subject must be notified as a minimum of the following:
	\begin{enumerate}[label=\alph*.]
		\item the \textbf{controller of the data file};
		\item the \textbf{purpose} of the processing;
		\item the \textbf{categories of data recipients} if a disclosure of data is planned.
	\end{enumerate}
\end{enumerate}

\subsection{Breach Of Obligations To Provide Information, To Register Or To Cooperate Art 34 DSG}

\begin{enumerate}[label=\arabic*]
	\item On complaint, \textbf{private persons are \underline{liable to a fine}} if they:
	\begin{enumerate}[label=\alph*.]
		\item breach their obligations under Articles 8–10 and 14, in that they \textbf{wilfully provide false or incomplete information}; or
		\item wilfully fail:
		\begin{enumerate}[label=\arabic*.]
			\item \textbf{to inform the data subject in accordance with Article 14} paragraph 1, or
			\item \textbf{to provide information required under Article 14} paragraph 2.
		\end{enumerate}
	\end{enumerate}
	\item Private persons are \textbf{\underline{liable to a fine} if they wilfully}:
		\begin{enumerate}[label=\alph*.]
			\item \textbf{fail to provide information} in accordance with Article 6 paragraph 3 or to declare files in accordance with Article11a or who in doing so wilfully provide false information; or
			\item \textbf{provide the Commissioner with false information} in the course of a case investigation (Art 29) or who refuse to cooperate.
	\end{enumerate}
\end{enumerate}

\subsection{Breach Of Professional Confidentiality Art 35 DSG}
\begin{enumerate}[label=\arabic*]
	\item Anyone who without authorisation \textbf{wilfully discloses confidential, sensitive personal data or personality profiles} that have come to their knowledge in course of their professional activities where such activities require the knowledge of such data, \textbf{on complaint, liable to a fine}.
\end{enumerate}

\subsection{Data Processing by Third Parties Art 10a DSG}
\begin{enumerate}[label=\arabic*]
	\item The processing of personal data may be assigned to third parties by agreement of by law if:
	\begin{enumerate}[label=\alph*.]
		\item the data is processed \textbf{only in the manner permitted for the instructing party itself}; and
		\item \textbf{it is not prohibited by a statutory or contractual duty of confidentiality}.
		\begin{enumerate}[label=\arabic*.]
			\item The instructing party must in particular \textbf{ensure that the third party guarantees data security}.
			\item Third parties may claim the same justification as the instructing party.
		\end{enumerate}
	\end{enumerate}
\end{enumerate}

\subsection{Employment and Data Protection Art 328b OR}
"The employer may data about the employee \textbf{only} process as it concerns \textbf{his qualification for the employment} or are \textbf{inevitable for the execution of the employment}. The regulations of the Swiss Data Protection Act are applicable."

\subsection{EU-Law Is Applicable for Swiss Companies}

\begin{itemize}
	\item EU-GDPR
	\item Directly applicable for all swiss companies if they address to customers in the EU  and are collecting personal data
\end{itemize}

\section{General Data Protection Regulation}
GDPR is directly applicable for Swiss companies if they
\begin{itemize}
	\item offer products and services in the EU/EWR and handle personal data for that purpose
	\item collect and exploit the personal data of website visitors from the EU
	\item send regular newsletter to recipients in the EU
	\item handle personal data by order of or as a concern or part of a company domiciled in the EU
\end{itemize}
EU supervisory authorities have the power to conduct inspections, to request relevant information, to give orders and, in cases of severe disregard of the GDPR, to levy fines up to €10-20 Mio. or up to 2-4\% of the worldwide annual revenue.

\begin{center}
	\includegraphics[width=0.9\linewidth]{img/gdpr_swiss_side}
\end{center}

\subsection{GDPR in Detail}
\begin{itemize}
	\item Augmenting of the peoples rights (Art 5/6 GDPR)
	\item Data storing only as long as it is necessary (storage limitation, Art 5 GDPR)
	\item Data protection by default by design (Art 25 GDPR)
	\item Big Data: Duty to do a Data protections impact assessment (Art 35 DSGVO)
	\item Duty to notify the supervisory authority of a personal data breach (Art 33 GDPR) and the data subject directly in cases of a high risk to the rights and freedoms of natural persons (Art 34 GDPR)
	\item Designation of a data protection officer (Art 37 GDPR). And if the requirements are met, a representative in the EU. (Art 27 Abs 1 DSGVO)
	\item When the processing is to be carried out on behalf of a controller: Only with sufficient guarantees, that is a contract (Art 28 Abs 1 GDPR)
	\item Right to data portability in a structured, commonly used and machine-readable format (Art 20 GDPR)
	\item Responsibility of the controller and duty to implement and document appropriate technical and organisational measures (Art 24 GDPR)
	\item No sub-sub-processing without written consent of the concerned party (Art 28 Abs 2 GDPR)
\end{itemize}

\subsection{Steps to Take}
\begin{center}
	\includegraphics[width=\linewidth]{img/gdpr_step_by_step_plan}
\end{center}
The base of any data protection audit is to assess the actual situation.
\begin{itemize}[noitemsep]
	\item What personal data does exist?
	\item In what form and where?
	\item For what reason?
	\item Who is responsible?
	\item Who has access?
	\item How long will be the data stored?
	\item How is the data protected technically and organisationally?
	\item What are the estimated possible risks of a data breach and their consequences?
\end{itemize}
The GDPR requires to adapt existing contracts, declarations (terms of use) and proceedings. Companies have an augmented duty to document. Appointing and naming a Data Protection Officer and, for Swiss companies, a representative located in the EU is mandatory.

Implement Technical \& Operational Measures, do a thorough and detailed documentation of data systems and their organisation.

\subsection{Cookies}
The European Court of Justice (ECJ) decided that explicit consent is required for cookies. The website operator is obliged to provide evidence.

\section{Trademark and Designs}
Trademarks influence consumer decisions every day. A strong trademark creates an identity, builds trust, distinguishes from competitors, and facilitates communication between producers and consumers. Registering a trademark gives the \textbf{exclusive right to use a certain sign for specific goods and services}. The usage of the trademark by a third party may be granted through licensing. A trademark owner can \textbf{prevent others} from \textbf{using an identical or similar sign for the same or similar goods and services}.

The owner of a trademark work is entitled to the exclusive and sole right to determine for
\begin{enumerate}[noitemsep,nosep]
	\item What
	\item When
	\item How
	\item By whom
\end{enumerate}
his or her trademark may be used.

\subsection{Special Type of Trademarks}
\begin{enumerate}
	\item Trademarks that have acquired distinctiveness through use\\
	Coca Cola, Virgin
	\item Trademarks that have become generic through use\\
	Trademarks can become designations for entire product classes through years of market presence and therefore lose their protective status, for example Xerox for printers
	\item Famous Trademarks
	\item Indication of Source\\
	Indications of source do not differentiate certain goods or services from each other by the manufacturer of the product as trademarks do, but by their geographical origin.
	
	If a sign contains distinctive components alongside a direct indication of source, it can be registered in the trademark register. They may only be used for products which originate from the indicated geographical area.
\end{enumerate}
The name of a business is not automatically protected as a trademark, thus it should be registered
\begin{enumerate}[noitemsep, nosep]
	\item in the commercial register
	\item in the trademark register
	\item as a domain name, which can also be registered as trademarks
\end{enumerate}

\subsection{Grounds for Refusal}
\begin{enumerate}
	\item Signs belonging to the public domain must remain available to everyone and cannot be registered. This includes, for instance, single letters or numbers or abbreviations.A sign may also not be purely descriptive of a characteristic, quality, type or place of production.\\
	Apple may not be used for a type of fruit, but it can be registered for a company selling computers
	\item A trademark may not be misleading or deceptive regarding properties such as source or quality.
	\item A trademark may not be offensive to moral standards or against the law.
\end{enumerate}

A trademark is only protected for those classes of goods and services for which it has been registered. If a trademark is not used within 5 years of registration, it may lose it's protection.

\subsection{Duration of Trademark Protection}
A trademark is protected once it is registered for a term of 10 years. This protection may be renewed multiple times, so a trademark that is in use can be protected indefinitely.

\subsection{Principle of Territoriality}
Trademarks are only valid in the country where they are registered. A trademark registered in Switzerland is therefore only protected in Switzerland. Outside the country where a trademark is registered, there is no protection, others may freely use the trademark!

\begin{enumerate}
	\item Applying directly to the country concerned
	\item The Madrid System\\
	Under the Madrid System trademark protection granted under Swiss law may be extended to other states or organisations that have signed the Madrid contract.
	\item EU Community Trademark
\end{enumerate}
The IGE does not verify whether identical or similar IP rights such as trademarks, company or domain names already exist. Potential conflicts with rights of other people are thus not examined in the registration process.

Checks before registering a trademark:
\begin{enumerate}
	\item Ensure that the sign is distinctive
	\item Search trademark registries
	\item Search commercial registers and domain names
\end{enumerate}

\subsection{Opposition or Cancellation}
\begin{enumerate}
	\item Opposition proceedings\\
	With publication on the Swiss register, the three-month opposition period begins and owners of prior identical or similar trademarks can file opposition to that mark during this period.
	\item Cancellation proceedings\\
	Any person is allowed to request the cancellation of a trademark registration on the grounds of non-use.
\end{enumerate}

\subsection{Design}
In the legal sense, design is understood to be the exterior form of a product or parts of it. It can be either two-dimensional or three-dimensional. Its form is characterised by the arrangement of lines, contours, colours and surfaces or by the material used.

Design appeals to our senses, evokes feelings, creates identity and distinguishes itself. This is why design has become one of the most crucial market factors and why counterfeiting is subsequently a frequent occurrence in this field.

Owners of a design right can prevent others from manufacturing, storing, offering, putting on the market, importing, exporting or transporting of such products as well as  being in possession of them.

Criteria for protection:
\begin{itemize}[label=-,noitemsep]
	\item The design is new, meaning no other identical or similar design has been published before application;
	\item The design is sufficiently different from existing designs in major characteristics.
\end{itemize}

A registered design is protected for a period of 5 years, this term can be extended by 5 years until a maximum of 25 years.

\end{document}
